%***************************************************
% abstract
\begin{abstract}
A two-bunch beam structure has been created and characterized for
plasma wakefield acceleration experiments at FACET. The longitudinal
inter-bunch spacing requirement for plasma wakefield acceleration is
on the order of the plasma wavelength, typically corresponding to a
range of 100--200\um. A high peak current is also a requirement, so
each bunch must be compact with high charge. To create the two-bunch
structure, a single chirped beam of 3.2\nC was dispersed and notched
with a tantalum finger collimator. The beam was then over-compressed,
creating a two-bunch ``drive and witness'' beam structure with
hundreds of \pC in each bunch and an inter-bunch spacing of 125\um. An
X-ray wiggler spectrometer was used to measure the notched energy
profile of the beam, and an X-band transverse deflecting structure was
used to measure the notched longitudinal profile of the beam.
\end{abstract}
