The chicane is symmetric about its center, with critical devices and diagnostics located at points where the ratio $\eta \delta / \sqrt{ \beta \varepsilon}$ is maximized.
%The longitudinal phase space evolves non-linearly as the beam is accelerated, chirped, and compressed, so that the final projected longitudinal emittance is much greater than $\varepsilon_{z0}$. $\varepsilon_{z0}$ remains an important quantity however, because it limits both the minimum achievable bunch length and 

% W-chicane, final focus, and notch collimator
Within the final 100\m of the FACET beamline there are several
critical tools and diagnostics that were used to generate and measure
our two-bunch beam structure. To create two bunches temporally
separated by only ${\like 500\fs}$, a chirped beam is sent from the
linac into the final chicane, wherein it is dispersed horizontally and
a notch is taken from its center with an adjustable tantalum
finger. Jaw collimators were used in addition to remove the far
extremeties of the beam, where the low and high energy tails sampled
too wide an expanse in $z$. Figure~\ref{fig:notch_col} shows a
simulation demonstrating the use of the custom collimators at
FACET. The remaining beam is then over-compressed by the end of the
chicane, and the notch in energy has become a notch in $z$. In
experiment it was found that about half of the initial beam had to be
collimated away to generate the cleanest two-bunch structure, leaving
${8.9 \pm XX \times 10^{10}}$ of the original ${2.2 \pm XX \times
10^{10}}$ electrons, as determined by simulations and current meters
along the beamline.

% sYAG
A wiggler spectrometer is located a point of equal dispersion compared
to the notch collimator according to the bilateral symmetry of the
chicane along the beam path. This device measures the dispersed charge
profile of the electron beam via the intensity pattern of X-rays
produced by the wiggling motion of the electron beam as observed on a
scintillating YAG crystal. This provides a direct measurement of the
transverse--and thus the spectral profile of the electron beam before
it reaches the plasma source. The wiggler spectrometer is a single
shot diagnostic, which allowed the experimenters to adjust and improve
the notched profile of the beam in live time. The longitudinally
correlated energy spread of the beam prior to notching was measured by
the wiggler spectrometer to be ${XX \pm \%}$. The energy spread of the
drive and witness bunches was measured to be ${XX \pm \%}$ and
${XX \pm \%}$, respectively. The resolution of the measurement is
determined by the point spread function of the YAG crystal, and the
CCD camera imaging setup. Figure~\ref{fig:sYAG} shows a typical
notched beam spectrum measured by the wiggler spectrometer from the
data.

% TCAV
A transverse deflecting waveguide structure occupies a location in the
chicane near the wiggler spectrometer and allows the experimenters to
measure the longitudial profile of the beam directly by streaking the
beam itself onto a scintillating YAG crystal or an optical transition
radiation (OTR) monitor screen at a location near the plasma
source. The streaking process precludes the beam from having an
optimal profile for PWFA, however, so this method of measuring the
longitudinal profile is limited to statistical studies of the beam
used in the experiment. The average longitudinal profile was estimated
by fitting two asymmetric gaussian distributions to a sample of 100
shots and taking the average of the fit parameters. The final numbers
for the two-bunch beam are listed in
Table~\ref{tab:two-bunch_beam}. The resolution of the measurement is
determined by the electron beam dynamics that govern the streaking of
the beam from the deflecting structure to the OTR screen, and the
point spread function of the YAG crystal, and the optical resolution
of the CCD camera imaging setup. Figure~\ref{fig:long_prof} shows a
typical notched longitudinal profile taken from the data using a YAG
crystal screen.

% Table: Experimental Parameters
\begin{table}[h]
\begin{tabular}{|c|c|c|c|c|}
  \hline
  \multicolumn{5}{|c|}{Drive Beam} \\
  \hline
  $Q$ (\pC) & $\sigma_{z,L}$ (\um) & $\sigma_{z,R}$ (\um) & $\sigma_r$ (\um) & $\mu_z$ (\um)\\
  \hline
  350       & 13                  & 27                   & 30               & 0   \\
  \hline
  \multicolumn{5}{c}{}\\
  \hline
  \multicolumn{5}{|c|}{Witness Beam} \\
  \hline
  $Q$ (\pC) & $\sigma_{z,L}$ (\um) & $\sigma_{z,R}$ (\um) & $\sigma_r$ (\um) & $\mu_z$ (\um) \\
  \hline
  580       & 42                  & 57                   & 30               & 126 \\
  \hline
\end{tabular}
\caption{
Experimental beam parameters for two-bunch FACET E200 PWFA experiment.
}
\label{tab:two-bunch_beam}
\end{table}

% two-bunch profile
For the data analyzed in this report, the drive bunch had a peak
current of approximately 2.1\kA, which corresponds to an expected
maximum plasma blowout radius of 76\um and a length of 150\um in the
preformed lithium plasma with density ${5.0 \times
10^{16}\invcc}$. The peak-to-peak distance between the drive and
witness bunch was 126\um, which put the core of the witness beam near
the peak of the accelerating field within the plasma wake. Because the
longitudinal extent of the witness beam was about 50\um rms, a
significant portion of the beam was located behind the rear edge of
the bubble in a region that is strongly
defocusing. Figure~\ref{fig:two-bunch_sim} shows a cutaway of the beam
and plasma density at a point midway through the plasma source from a
simulation run in QuickPIC~\ref{QuickPIC}. The loaded longitudinal
field profile on axis as determined by the simulation is superimposed
on the denstiy profiles and illustrates the loading of the wake from
the witness bunch used in the experiment.
